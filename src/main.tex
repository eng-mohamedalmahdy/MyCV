%%%%%%%%%%%%%%%%%
% This is an sample CV template created using altacv.cls
% (v1.3, 10 May 2020) written by LianTze Lim (liantze@gmail.com). Now compiles with pdfLaTeX, XeLaTeX and LuaLaTeX.
% This fork/modified version has been made by Nicolás Omar González Passerino (nicolas.passerino@gmail.com, 15 Oct 2020)
%
%% It may be distributed and/or modified under the
%% conditions of the LaTeX Project Public License, either version 1.3
%% of this license or (at your option) any later version.
%% The latest version of this license is in
%%    http://www.latex-project.org/lppl.txt
%% and version 1.3 or later is part of all distributions of LaTeX
%% version 2003/12/01 or later.
%%%%%%%%%%%%%%%%

%% If you need to pass whatever options to xcolor
\PassOptionsToPackage{dvipsnames}{xcolor}

%% If you are using \orcid or academicons
%% icons, make sure you have the academicons
%% option here, and compile with XeLaTeX
%% or LuaLaTeX.
% \documentclass[10pt,a4paper,academicons]{altacv}

%% Use the "normalphoto" option if you want a normal photo instead of cropped to a circle
% \documentclass[10pt,a4paper,normalphoto]{altacv}

\documentclass[10pt,a4paper,ragged2e,withhyper]{altacv}

%% AltaCV uses the fontawesome5 and academicons fonts
%% and packages.
%% See http://texdoc.net/pkg/fontawesome5 and http://texdoc.net/pkg/academicons for full list of symbols. You MUST compile with XeLaTeX or LuaLaTeX if you want to use academicons.

% Change the page layout if you need to
\geometry{left=1.2cm,right=1.2cm,top=1cm,bottom=1cm,columnsep=0.75cm}

% The paracol package lets you typeset columns of text in parallel
\usepackage{paracol}
\usepackage{blindtext}
\usepackage{multicol}

% Change the font if you want to, depending on whether
% you're using pdflatex or xelatex/lualatex
\ifxetexorluatex
% If using xelatex or lualatex:
\setmainfont{Roboto Slab}
\setsansfont{Lato}
\renewcommand{\familydefault}{\sfdefault}
\else
% If using pdflatex:
\usepackage[rm]{roboto}
\usepackage[defaultsans]{lato}
% \usepackage{sourcesanspro}
\renewcommand{\familydefault}{\sfdefault}
\fi

% ----- MONO MODE -----
%\definecolor{SlateGrey}{HTML}{2E2E2E}
%\definecolor{LightGrey}{HTML}{666666}
%\definecolor{PrimaryColor}{HTML}{000000}
%\definecolor{SecondaryColor}{HTML}{000000}
%\definecolor{ThirdColor}{HTML}{000000}
%\colorlet{name}{PrimaryColor}
%\colorlet{tagline}{PrimaryColor}
%\colorlet{heading}{PrimaryColor}
%\colorlet{headingrule}{ThirdColor}
%\colorlet{subheading}{SecondaryColor}
%\colorlet{accent}{SecondaryColor}
%\colorlet{emphasis}{SlateGrey}
%\colorlet{body}{LightGrey}
% Define your branding colors

\definecolor{PrimaryColor}{HTML}{2E675F}  % Main text and headings
\definecolor{SecondaryColor}{HTML}{0C0A49}  % Subheadings and accents
\definecolor{TertiaryColor}{HTML}{983D0C}   % Highlight or accent color
\definecolor{QuaternaryColor}{HTML}{9B206F} % Alternative accent color
\definecolor{BackgroundColor}{HTML}{F2F2F2}

\colorlet{name}{PrimaryColor}
\colorlet{tagline}{PrimaryColor}
\colorlet{heading}{PrimaryColor}
\colorlet{headingrule}{PrimaryColor}
\colorlet{subheading}{SecondaryColor}
\colorlet{accent}{TertiaryColor}
\colorlet{emphasis}{SecondaryColor}
\colorlet{body}{LightGrey}  % Assuming you want to keep LightGrey as the background color



\pagecolor{BackgroundColor}


% Change some fonts, if necessary
\renewcommand{\namefont}{\Huge\rmfamily\bfseries}
\renewcommand{\personalinfofont}{\small\bfseries}
\renewcommand{\cvsectionfont}{\large\rmfamily\bfseries}
\renewcommand{\cvsubsectionfont}{\large\bfseries}

% Change the bullets for itemize and rating marker
% for \cvskill if you want to
\renewcommand{\itemmarker}{{\small\textbullet}}
\renewcommand{\ratingmarker}{\faCircle}

%% sample.bib contains your publications
%% \addbibresource{sample.bib}
% \documentclass[12pt]{scrartcl}
\usepackage{lipsum}
\usepackage{wasysym}
%\usepackage{lstdoc}
%
%
\usetikzlibrary{calc}
%
\begin{document}
% \begin{tikzpicture}[overlay,remember picture]

%         \draw [xshift=4mm,line width=1pt,rounded corners=16pt]
%         ($ (current page.north west) + (.5cm,-.5cm) $)
%         rectangle
%         ($ (current page.south east) + (-.5cm,.5cm) $);

% \end{tikzpicture}
%
    \name{Mohamed Almahdy}
    \tagline{Mobile Applications Developer}
    %% You can add multiple photos on the left or right

    \personalinfo{
        \email{engmohamedalmahdy@gmail.com}\smallskip
        \mobile{+201096475710}
        \location{Nasr City, Cairo}
        \youtube{LightFeather42}
        \linkedin{https://www.linkedin.com/in/eng-mohamed-almahdy}
        \github{https://github.com/eng-mohamedalmahdy}

    %\medium{nicolasomar}
    %% You MUST add the academicons option to \documentclass, then compile with LuaLaTeX or XeLaTeX, if you want to use \orcid or other academicons commands.
    % \orcid{0000-0000-0000-0000}
    %% You can add your own arbtrary detail with
    %% \printinfo{symbol}{detail}[optional hyperlink prefix]
    % \printinfo{\faPaw}{Hey ho!}[https://example.com/]
    %% Or you can declare your own field with
    %% \NewInfoFiled{fieldname}{symbol}[optional hyperlink prefix] and use it:
    % \NewInfoField{gitlab}{\faGitlab}[https://gitlab.com/]
    % \gitlab{your_id}
    }

    \makecvheader
    %% Depending on your tastes, you may want to make fonts of itemize environments slightly smaller
    % \AtBeginEnvironment{itemize}{\small}

    %% Set the left/right column width ratio to 6:4.


    % ----- ABOUT ME -----
    \cvsection{Profile}
    \begin{quote}

        \textbf{Mid-level}  Mobile Apps Engineer with \textbf{2+ years of hands-on experience},
        adept in \textbf{Kotlin, Java \& Swift}.
        Created \& Contributed at \textbf{10+ mobile Apps}.
        My primary are in \textbf{Native Android}, \textbf{iOS} and \textbf{Kotlin Multiplatform} frameworks.
        Possess a \textbf{solid foundation} in \textbf{digital marketing}, adding versatility to my tech expertise.
    \end{quote}
    % ----- ABOUT ME -----


    \cvsection{Skills}
    {\large\begin{itemize}
               \item Strong grasp of \textbf{Object-Oriented Programming (OOP)} principles, \textbf{data structures}, and \textbf{algorithms}.
               \item Proficient in \textbf{applying} and \textbf{using} common \textbf{design patterns}, \textbf{SOLID principles}, and best practices.
               \item Solid understanding of \textbf{HTTP Requests} and working with \textbf{RESTful APIs}.
               \item Expertise in version control systems such as \textbf{Git} and \textbf{GitHub}.
               \item Proficient in implementing clean architecture patterns such as \textbf{MV(C, VM)} and \textbf{modularization}.
               \item Skilled in \textbf{multithreading}, using built-in and external libraries like \textbf{Kotlin Coroutines} and \textbf{Swift GCD}.
               \item Good knowledge of other programming paradigms such as \textbf{Functional} and \textbf{Reactive Programming}.
               \item Proficient in applying \textbf{Dependency Injection} principles.
               \item Experienced in creating and executing thorough \textbf{unit testing} procedures.
    \end{itemize}}

    \columnratio{0.333334}

    % Start a 2-column paracol. Both the left and right columns will automatically
    % break across pages if things get too long.



    \begin{paracol}{2}

        % ----- LEARNING -----


        \cvsection{Programming Languages}
        \cvtag{Kotlin}
        \cvtag{Java}
        \cvtag{Swift}

        \cvsection{Frameworks}
        \cvtag{Native Android}
        \cvtag{Kotlin Multiplatform}
        \cvtag{Native iOS}

        \cvsection{Android Tools}
        \cvtag{Kotlin Compose}
        \cvtag{Coroutines}
        \cvtag{Retrofit}
        \cvtag{Rx Java \& Kotlin}
        \cvtag{Image Loaders}
        \cvtag{Koin}
        \cvtag{Room}
        \cvtag{Realm}
        \cvtag{Dagger hilt}
        \cvtag{JUnit}
        \cvtag{MocKK}
        \cvtag{Mockito}

        \cvsection{iOS Tools}
        \cvtag{UIkit}
        \cvtag{Storyboard}
        \cvtag{Auto layout}
        \cvtag{Swift UI}
        \cvtag{GCD}
        \cvtag{RX Swift \& Cocoa}
        \cvtag{Core Data}
        \cvtag{Alamofire}
        \cvtag{Kingfisher}


        % ----- LEARNING -----

        \newpage
%         \begin{tikzpicture}[overlay,remember picture]

%         \draw [xshift=4mm,line width=1pt,rounded corners=16pt]
%         ($ (current page.north west) + (.5cm,-.5cm) $)
%         rectangle
%         ($ (current page.south east) + (-.5cm,.5cm) $);

% \end{tikzpicture}
        %% Switch to the right column. This will now automatically move to the second
        %% page if the content is too long.
        \switchcolumn


        % ----- EXPERIENCE -----
        \cvsection{Experience}

        \cvevent{Mid Level Android developer  }{| \href{https://www.linkedin.com/company/dtag/}{\textbf{DTag}}}{04/2023 -- Now}{Cairo, Egypt}
        \begin{itemize}
            \item Created \& Contributed at 3 Mobile Apps
        \end{itemize}
        \divider

        \cvevent{JR. Software developer  }{| \href{https://www.linkedin.com/company/virtualworkernow/} {\textbf{Virtual Worker Now}}}{12/2021 -- 5/2022}{Cairo, Egypt}
        \begin{itemize}
            \item 3 Websites contributions
            \item 2 Mobile Apps contributions
        \end{itemize}
        \divider

%        \cvevent{Teaching }{| \textbf{Light Feather}}{2022 - NOW}{}
%        \begin{itemize}
%            \item  \href{https://www.youtube.com/c/LightFeather42}{YouTube Channel}
%        \end{itemize}
%        \divider

        \cvevent{JR. Android developer }{| \href{https://www.linkedin.com/company/elmnassa/}{\textbf{ElMnassa Innovation}}}{6/2020 -- 12/2020}{Cairo, Egypt}
        \begin{itemize}
            \item 2 Mobile Apps contributions
        \end{itemize}

        % ----- EXPERIENCE -----

        % ----- EDUCATION -----
        \cvsection{Education}
        \cvevent{B.S Computer Engineering}{| \textbf{Al-Azhar University.}}{2017 -- 2021}{Cairo, Egypt}
        \begin{itemize}
            \item Graduation Project: \textbf{Zakerly ( E-learning Mobile App )} Grade : \textbf{Excellent}
        \end{itemize}


        % ----- EDUCATION -----

        % ----- Courses -----

        \begin{multicols}{2}
            \cvsection{Recent Courses}
            \cvevent
            {\faApple  iOS using SwiftUI} {| \textbf{Stanford}}{}{}
            \divider


            \cvevent
            {\faAndroid Android Development}{| \textbf{Udacity}}{}{}
            \divider

            \cvevent
            {\faEdit 5.95J Teaching Course}{| \textbf{MIT}}{}{}


            \columnbreak
            \cvsection{Recent Books Readings}
            \cvevent {\faBook Clean Mobile Architecture}{}{}{}
            \cvevent {\faBook Clean Architecture}{}{}{}
            \cvevent {\faBook Algorithms Unplugged}{}{}{}
            \cvevent {\faBook Head First OOA/D}{}{}{}
            \cvevent {\faTrello  \href{https://trello.com/b/pkZlgQVu/my-readings-board}{\textbf{All My readings book list: \textcolor{SecondaryColor}{HERE}}}}{}{}{}

        \end{multicols}

        % ----- Courses -----
    \end{paracol}
    \begin{paracol}{1}
        \cvsection{Personal Projects}

        \cvevent{\textbf{Samoolah}}{\cvrepo{|\faLock\faAndroid}{}}{}{}
        \begin{itemize}
            \item \textbf{Services Provider} App With 2 Distributions 1 For the Client 1 For the Technician
            \item \textbf{Used Technologies} : MVVM with \textbf{Modularization} - \textbf{Compose} - Rest APIs with Retrofit \& Gson Serialization - Navigation Component - Kotlin Coroutines with Flows
        \end{itemize}
        \divider

        \cvevent{\textbf{Islami 180}}
        {\cvrepo{|\faGooglePlay\faAndroid}
        {https://play.google.com/store/apps/details?id=com.lightfeather.islami180}}{}{}
        \begin{itemize}
            \item \textbf{}Islamic application contains:
            Quran + Quran Tafseer - Prayer Azan - Qibla Finder - Azkar - Hijri Calendar - Tasbeeh
            \item \textbf{Used Technologies}: \textbf{MVVM -
            Room database (for Quarn data persistence) -
            Retrofit (for prayer azan API requests) -
            Android navigation component -
            Json to Gson serialization -
            Work Manager -
            Notifications Manager -
            Alarm Manager
            Senesor Manager And more
            }
        \end{itemize}
        \divider

        \cvevent{\textbf{Violet}}
        {\cvrepo{|\faGithub\faAndroid\faGooglePlay}
        {https://play.google.com/store/apps/details?id=com.lightfeather.violet}}{}{}
        \begin{itemize}
            \item Violet is a secret notes app where your notes can be accessed view the title of the note or 2FA method Your notes will also be stored in encrypted database so no one but you can see your notes
            \item \textbf{Used Technologies}: \textbf{MVVM -
            Live data -
            Dagget hilt -
            Room database (for Notes data persistence) -
            Kotlin kotlin coroutines -
            Encrypted shared preferences -
            Android navigation component -
            Json to Gson serialization -
            Android Guide view -
            Android lottie library -
            Android localization library -
            Android SDP library}
        \end{itemize}
        \divider

        \cvevent{\textbf{Friends Keep}}
        {\cvrepo{|\faGithub\faAndroid\faGooglePlay}
        {https://play.google.com/store/apps/details?id=com.lightfeather.friendskeep}}{}{}
        \begin{itemize}
            \item \textbf{Friends Keep is an app where you add your favorite friends and every thing you need to remember about them and it will notify you with important dates like birth date}
            \item \textbf{Used Technologies} : \textbf{MVVM -
            Kotlin coroutines -
            Koin -
            Work manager -
            Notification manager -
            Room for data persistence -
            Stored the images as compressed Base64 string -
            Google ads SDK -
            Live data -
            Navigation component -
            Lotti animations
            Circular image view}
        \end{itemize}
        \divider

        \cvevent{\textbf{Pet Matrices}}
        {\cvrepo{|\faLock\faApple}
        {}}{}{}
        \begin{itemize}
            \item \textbf{}Pet Matrices is a lifestyle application for your pets where you can record and track your pets activities and given calories, you also can record and track their health condition and get some nice advice from an expert
            \item \textbf{Used Technologies} : \textbf{SwiftUI, REST API, MVVM archticture pattern, Swift GCD}
        \end{itemize}
        \divider
        \cvevent{\textbf{Zakerly}}
        {\cvrepo{|\faGithub\faAndroid}
        {https://github.com/eng-mohamedalmahdy/Zakerly}}{}{}
        \begin{itemize}
            \item \textbf{}Zakerly is an app used to connect students with teachers to learn online as an E-learning service And helps them attending online sessions individuals or groups
            \item \textbf{Used Technologies}: \textbf{MVVM
            Firebase Kit (real time database \& cloud messaging)
                Realm database (for data persistence)
                Retrofit (for FCM API requests)
                Android navigation component
                Json to Gson serialization
                Jitsi meet sdk}
        \end{itemize}
        \divider

        \cvevent{\textbf{MusicHub}}
        {\cvrepo{|\faGithub\faAndroid}
        {https://github.com/eng-mohamedalmahdy/MusicHub}
        }{}{}
        \begin{itemize}
            \item \textbf{} Music Player application with YouTube download
            \item Technologies uses: \textbf{MVVM - Live data - Dagger hilt - SQLite - coroutines - Kotlin Compose - Flows -\linebreak Foreground services - Custom notifications}
        \end{itemize}
        \divider


        \cvevent{\textbf{GigaMan}}
        {\cvrepo{|\faGithub\faGamepad}
        {https://github.com/eng-mohamedalmahdy/GigaMan}}{}{}
        \begin{itemize}
            \item \textbf{}A platformer game based on the mega man game built using libgdx
            \item \textbf{Used Technologies}: \textbf{Libgdx}
        \end{itemize}
        \divider

        \cvevent{\textbf{Memorise}}
        {\cvrepo{|\faGithub\faApple}
        {https://github.com/eng-mohamedalmahdy/Memorise}
        }{}{}
        \begin{itemize}
            \item \textbf{} Memorising game created with swift and Swift UI. \linebreak
            A project for Stanford CS193p course at : https://cs193p.sites.stanford.edu/
        \end{itemize}
        \divider

        \begin{center}
            \textbf{\Large \href{https://github.com/eng-mohamedalmahdy}{More Projects at my GitHub}}
            {\cvrepo{\faGithub}
            {https://github.com/eng-mohamedalmahdy}}{}{}
        \end{center}
        \divider
    \end{paracol}

\end{document}
